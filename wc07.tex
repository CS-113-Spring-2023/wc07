\documentclass[a4paper]{exam}

\usepackage{amsmath}
\usepackage{amssymb}
\usepackage{amsthm}
\usepackage{array}
\usepackage{geometry}
\usepackage{hyperref}
\usepackage{titling}

\newcolumntype{C}{>{$}c<{$}} % math-mode version of "c" column type

\newcommand\mbb[1]{\ensuremath{\mathbb{#1}}}

\runningheader{CS/MATH 113}{WC07: Sets}{\theauthor}
\runningheadrule
\runningfootrule
\runningfooter{}{Page \thepage\ of \numpages}{}

\printanswers

\title{Weekly Challenge 07: Sets\\CS/MATH 113 Discrete Mathematics}
\author{team-name}  % <== for grading, replace with your team name, e.g. q1-team-420
\date{Habib University | Spring 2023}

\qformat{{\large\bf \thequestion. \thequestiontitle}\hfill}
\boxedpoints

\begin{document}
\maketitle

\begin{questions}

\titledquestion{Distributing over a Collection of Sets}
  Imagine a collection, $S$, of sets. $S$ may be finite, i.e. $\exists n\in \mbb{Z}^*\ni S = \{S_0,S_1,S_2, S_3, \ldots S_n\}$, or infinite, i.e., $S = \{S_0,S_1,S_2, S_3, \ldots \}$.

  $S$ can be represented with the help of an \textit{index set}, $A$, as $S = \{S_i\mid i \in A\}$. When $S$ is finite, then $A\subset \mbb{Z}^*$, otherwise $A=\mbb{Z}^*$.

  Union and intersection involving collections of sets are defined as follows.
  \begin{align*}
    \bigcup_{i \in A}S_i = \{x\mid \exists i \in A \ni x \in S_i\}\\
    \bigcap_{i \in A}S_i = \{x\mid\forall i \in A\; (x \in S_i)\}
  \end{align*}
  \begin{parts}
  \part[5] Show that
    \[
      \bigcup_{i\in A}(T \cap S_i) = T \cap \left(\bigcup_{i\in A}S_i  \right).
    \]
  \part[5] Show that
    \[
      \bigcap_{i\in A}(T \cup S_i) = T \cup \left(\bigcap_{i\in A}S_i  \right).
    \]
  \end{parts}
  
  \begin{solution}
      % Enter your solution here.
    \end{solution}

\end{questions}

\end{document}

%%% Local Variables:
%%% mode: latex
%%% TeX-master: t
%%% End:
